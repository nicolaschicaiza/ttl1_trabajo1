%Preamble
\pdfminorversion=4
\documentclass[11pt]{article}
\usepackage[spanish]{babel}
\usepackage[utf8x]{inputenc}
\usepackage{tabularx} % extra features for tabular environment
\usepackage{amsmath}  % improve math presentation
\usepackage{graphicx} % takes care of graphic including machinery
\usepackage{geometry} % decreases margins
\usepackage{cite} % takes care of citations
\usepackage[final]{hyperref} % adds hyper links inside the generated pdf file
\usepackage{booktabs}
\usepackage{subcaption}
\usepackage{fancyhdr}
\usepackage{parskip}
\usepackage{amssymb, amsmath} % Paquetes matemáticos de la American Mathematical Society
\usepackage{float}
\usepackage{multirow}
\usepackage[all]{xy}
\usepackage{tikz}
\usetikzlibrary{matrix}
\usetikzlibrary{calc}
\usetikzlibrary{fit}
%\usepackage{showframe}

\geometry{
    papersize = {216mm, 279mm},
    width = 16.6cm,
    height = 25cm,
    headsep = 5mm,
    head = 2.8cm,
    marginpar = 5mm,
    includeall,
}

\fancyhf{}
\renewcommand{\headrulewidth}{0pt}
\fancyhead[LO,LE]{
    \begin{minipage}{3cm}
        \includegraphics[width=0.7\textwidth]{Escudo.jpg}
    \end{minipage}
}
\fancyhead[RO,RE]{
    \textsf{
        Serie de Fourier aplicado a una función Diente de Sierra\\
        Teoría de telecomunicaciones I, Grupo A12\\
        \date{\today}.   
    }
}
\fancyfoot[C]{\thepage}

\pagestyle{fancy}

\hypersetup{
	colorlinks=true,       % false: boxed links; true: colored links
	linkcolor=black,        % color of internal links
	citecolor=black,        % color of links to bibliography
	filecolor=magenta,     % color of file links
	urlcolor=blue         
}
\spanishdecimal{.}

%++++++++++++++++++++++++++++++++++++++++
%Content

\title{\textbf{Series de Fourier}}
\author{
    Jefry Nicolás Chicaiza - \url{jefryn@unicauca.edu.co}\\
    Jose Nicolás Zambrano - \url{jnzambranob@unicauca.edu.co}
}
\date{}

\begin{document}
\maketitle
\thispagestyle{fancy}
\section*{Introducción}
    En el siguiente documento se desarrollará el informe del Trabajo 1 de la asignatura 
    Teoría de las telecomunicaciones 1. El trabajo presenta inicialmente el desarrollo 
    analítico mediante serie de Fourier de la señal planteada, la cual es del tipo 
    "diente de sierra"  trasladado en el tiempo.\\
    \\
    Iniciar con el desarrollo analítico es necesario debido a que para alcanzar los 
    resultados esperados en la simulación, se requiere conocer de antemano los coeficientes 
    de la serie de Fourier que permitirán reconstruir la señal a través de iteraciones 
    realizadas con MATLAB.\\
    \\
    Posteriormente se abordaran las hipótesis planteadas en el documento guiá del trabajo 
    y se buscará llegar a conclusiones y síntesis a partir de los datos obtenidos en la 
    simulación de los diferentes escenarios.\\
    \\
    Los teoremas e hipótesis nos dicen que la serie de Fourier de cualquier señal periódica de potencia 
    finita la podemos obtener por medio de una suma infinita de funciones sinusoidales. La serie de 
    Fourier de una señal pueden expresarse de dos maneras, una representada por serie trigonométrica y 
    otra con representación de serie compleja.\\ 
    \\
    En este documento unicamente se realiza los cálculos de la serie de Fourier para la señal propuesta 
    con la representación trigonométrica, que observamos a continuación:

    \begin{equation}
        x(t)=a_{0}+\sum_{n=1}^{\infty} a_{n}cos(2\pi nf_{0}t)+b_{n}sin(2\pi f_{0}t)
        \label{equation1}
    \end{equation}
    
    Donde los términos $a_{0}$, $a_{n}$ y $b_{n}$ son los coeficientes de la serie de Fourier, de los cuales
    es necesario realizar cálculos para encontrar sus valores y lograr representar la serie de la señal dada.\\
    \\
    El desarrollo de este documento esta constituido por una sesión que brinda información 
    de como se obtuvo las diferentes expresiones y el plan de pruebas, que hacen posible la 
   
\section*{Metodología}
    La metodología empleada para el desarrollo de la serie de Fourier a la señal planteada se 
    logro mediante la aplicación de los teoremas e hipótesis, en el apartado anterior se menciono 
    la necesidad de calcular los valores de ciertos términos para logra obtener la representación 
    matemática de la serie de Fourier de la señal planteada.\\ 
    \\
    En primer lugar es importante conocer el comportamiento que nos describe los términos, el primer 
    termino de la serie, $a_{0}$, no está asociado con una frecuencia y es una constante, además es 
    conocido como el nivel DC de la señal, representa el cambio de la señal sobre un nivel arbitrario 
    de referencia. Este termino lo podemos calcular de la siguiente manera:

    \begin{equation}
        a_{0}=\frac{1}{T} \int_{T}^{} x(t)dt
        \label{equation2}
    \end{equation}
    
    De la expresión anterior podemos deducir que el termino DC representa el área bajo la curva de la 
    señal \cite{diapositivas}, que hasta el momento no conocemos la función que la representa, por tanto es necesario obtener 
    la función a partir de su gráfica. En la figura \ref{figure1} observamos la señal asignada.

    \begin{figure}[H]
        \centering
        \includegraphics[width=0.5\textwidth]{img/figure1.png}
        \caption{Gráfica de la señal tipo "diente de sierra".}
        \label{figure1}
    \end{figure}

    En \cite{Sawtooth} se menciona que durante el intervalo en el que se presenta la señal, la función que 
    responde a una onda diente de sierra se da de la siguiente manera:
    
    \begin{equation*}
        x(t)=\frac{A}{T} t
    \end{equation*}

    Por lo que la función a la que responde la señal que se plantea debe ser del mismo modo, la expresión 
    de la señal es la siguiente:

    \begin{equation}
        x(t)=\frac{1}{4} t+\frac{1}{8}
        \label{equation3}
    \end{equation}

    Ahora que conocemos la expresión de la señal se procederá a obtener el valor del termino $a_{0}$, con las 
    ecuaciones \ref{equation2} y \ref{equation3}:

    \begin{equation*}
        a_{0}=\frac{1}{4} \int_{-\frac{1}{2}}^{\frac{7}{2}} \frac{1}{4} t+\frac{1}{8}
    \end{equation*}
    \begin{equation}
        a_{0}=\frac{1}{2}
        \label{equation4}
    \end{equation}    

    Para el valor de los términos restantes es importante mencionar que con frecuencia las simetrías simplifican 
    a los problemas matemáticos. En el caso de la series de Fourier, se utiliza la simetría en la paridad para 
    simplificar el problema. En el caso de funciones pares el desarrollo de la serie de Fourier implica solamente 
    la necesidad de calcular $a_{0}$ y $a_{n}$, y sólo $b_{n}$ para impares \cite{parimpar}.\\
    \\
    La señal que concierne a este texto podemos deducir de su gráfica(figura \ref{figure1}) que no se comporta igual que su imagen respecto 
    al eje $y$, ni tampoco hay presencia de simetría al rotar la gráfica en 180 grados. Por tanto, se trata de una 
    función sin paridad, lo que implica la necesidad de calcular todos los términos de la serie trigonométrica de Fourier .\\
    \\
    Las siguientes expresiones corresponden al calculo de los coeficientes $a_{n}$ y $b_{n}$:

    \begin{equation}
        a_{n}=\frac{2}{T} \int_{T} x(t)cos(2\pi nf_{0}t)dt
        \label{equation5}
    \end{equation}
    \begin{equation}
        b_{n}=\frac{2}{T} \int_{T} x(t)sin(2\pi nf_{0}t)dt
        \label{equation6}
    \end{equation}

    los resultados obtenidos al realizar los cálculos para los coeficientes son los siguientes (los cálculos que 
    se presentan en este documento están simplificados, por tal motivo se integra el paso a paso en el apartado de 
    Anexos):\\

    \begin{itemize}
        \item Calculo $a_{n}$
            \begin{equation*}
                a_{n}=\frac{1}{8}\int_{-\frac{1}{2}}^{\frac{7}{2}} tcos(2\pi nf_0t)dt + 
                \frac{1}{16} \int_{-\frac{1}{2}}^{\frac{7}{2}} cos(2\pi nf_0t)dt
            \end{equation*}
            \begin{equation}
                a_{n}=\frac{1}{\pi n} sin\left(\frac{7\pi n}{4}\right)+\frac{1}{2\pi^2 n^2}\left(cos\left(\frac{7\pi n}{4}\right)-cos\left(\frac{\pi n}{4}\right)\right)
                \label{equation7}
            \end{equation}
        \item Calculo $b_{n}$
            \begin{equation*}
                b_{n}=\frac{1}{8}\int_{-\frac{1}{2}}^{\frac{7}{2}} tsin(2\pi nf_0t)dt + 
                \frac{1}{16} \int_{-\frac{1}{2}}^{\frac{7}{2}} sin(2\pi nf_0t)dt
            \end{equation*}
            \begin{equation}
                b_{n}=-\frac{1}{\pi n} cos\left(\frac{7\pi n}{4}\right)+\frac{1}{2\pi^2 n^2}\left(sin\left(\frac{7\pi n}{4}\right)+sin\left(\frac{\pi n}{4}\right)\right)
                \label{equation8}
            \end{equation}
    \end{itemize}
    
    Lo siguiente sera completar la serie de Fourier para obtener la representación matemática de las funciones sinusoidales 
    que construyen la señal planteada, reemplazamos las ecuaciones \ref{equation7} y \ref{equation8} en la ecuación 
    \ref{equation1}:

    \begin{multline}
        x(t)=\frac{1}{2}+\sum_{n=1}^{\infty} \left[\frac{1}{\pi n} sin\left(\frac{7\pi n}{4}\right)+\frac{1}{2\pi^2 n^2}\left(cos\left(\frac{7\pi n}{4}\right)-cos\left(\frac{\pi n}{4}\right)\right)
        \right]cos\left(\frac{\pi nt}{2}\right)...\\ +\left[-\frac{1}{\pi n} cos\left(\frac{7\pi n}{4}\right)+\frac{1}{2\pi^2 n^2}\left(sin\left(\frac{7\pi n}{4}\right)+sin\left(\frac{\pi n}{4}\right)\right)
        \right]sin\left(\frac{\pi nt}{2}\right)
        \label{equation9}
    \end{multline}

    Ahora que se ha obtenido la expresión que representa la serie de Fourier de la señal, se procederá a realizar 
    un análisis de la serie de Fourier a través de una simulación realizada en computadora, en ella podremos 
    visualizar como se resuelve para cientos, o incluso miles, de componentes. Podemos comprobar qué tan buena 
    es la representación reconstruyendo la señal utilizando diferentes cantidades de componentes, para esto 
    se realizaran pruebas con un Script desarrollado en MATLAB.\\
    \\
    Después de haber realizado el planteamiento de los fundamentos matemáticos requeridos para solucionar 
    el problema, se procedió a realizar la planificación del diseño practico requerido para solucionar 
    el mismo.\\
    \\
    Se pueden identificar 6 objetivos clave a desarrollar con la simulación en MATLAB:\\

    \begin{enumerate}
        \item Reconstrucción de la señal dada a partir de la serie de Fourier.
        \item Justificación del número de coeficientes necesarios para ua reconstrucción adecuada de la 
            señal original.
        \item Análisis del espectro de magnitud cuando el periodo T de la señal reconstruida cambia.
        \item Análisis del espectro de magnitud cuando se agrega espacio de tiempo donde $x(t)=0$ entre 
            los pulsos periódicos "diente de sierra".
        \item Análisis de la continuidad del número de coeficientes necesarios cuando el periodo de la 
            señal cambia.
        \item Análisis de la continuidad del número de coeficientes necesarios cuando se agregan ceros 
            como en el objetivo 4.
    \end{enumerate}

    Después de realizar el análisis de los requerimientos anteriores, se plantea el siguiente \underline{esquema}
    \underline{general} para el desarrollo de la simulación:
    \\
    \begin{center}
         \xymatrix{
            % primera fila
            *++[F-,]\txt{1. Declaración de variables y espacios \\vectoriales  a utilizar} \ar[d] & &
            *++[F-,]\txt{4. Criterios utilizados para justificar el \\numero de coeficientes necesarios para \\ 
                reconstruir la señal} \ar{d}
            \\
            % segunda fila
            *++[F-,]\txt{2. Calculo de los coeficientes de Fourier} \ar[d] & &
            *++[F-,]\txt{5. Graficación de señales y espectro de la \\serie de Fourier}
            \\
            % tercer fila
            *++[F-,]\txt{3. Reconstrución de la señal a partir del \\calculo de la Serie de Fourier}
            \ar`r[ruu]`[rruu][rruu] % \ar[uuur]
        }    
    \end{center}

    Para la realización del script de simulación se utilizó el paradigma de programación estructurada, ya 
    que el mismo permite agilidad en el desarrollo del código, así como también facilita el desarrollo de los 
    planteamientos matemáticos necesarios para lograr los objetivos de la simulación.\\
    \\
    Adicionalmente la programación estructurada, brinda al observador del código una perspectiva secuencial en 
    el desarrollo del algoritmo de simulación, y, por lo tanto, un orden lógico y trazable de los resultados 
    esperados en la ejecución de este.\\
    
    \textbf{Plan de pruebas}\\
    \\
    Retomando los 6 objetivos clave a desarrollar con la simulación; se plantea el siguiente plan de pruebas 
    para cadaa uno de ellos:

    \begin{table}[h!]
        \begin{center}
            \begin{tabular}{| m{5cm} | m{10cm} |}
                \hline
                \textbf{Objetivo Clave} & \textbf{Pruebas y Criterio de Satisfacción} \\ \hline
                \multirow{3}{4.5cm}{Reconstrucción de la señal} 
                    & Se dibuja mediante la simulación la señal original y la señal reconstruida. \\ \cline{2-2} 
                    & Se observan similitudes y diferencias. \\ \cline{2-2}
                    & Se concluye si las señales son gráficamente similares o diferentes. \\ \hline
                \multirow{4}{4.5cm}{Justificación del número de Coeficientes}
                    & Se calcula la igualdad de Parseval. \\ \cline{2-2} 
                    & Se observa el fenómeno de Gibbs en la vecindad delas desigualdades de la señal. \\ \cline{2-2}
                    & Se analizan los valores obtenidos con 5 diferentes números de armónicos. \\ \cline{2-2}
                    & Se concluye el valor de armónicos necesarios para tener una reconstrucción 
                    aceptable de la señal. \\ \hline
                \multirow{3}{4.5cm}{Análisis del espectro de magnitud cuando el periodo $T$ cambia}
                    & Se dibuja y se analiza el espectro de magnitud con periodo $T=4$ y $N=100$. \\ \cline{2-2}
                    & Se dibuja y se analiza el espectro de magnitud con el mismo $N$ y 3 valores diferentes 
                    de periodo $T$. \\ \cline{2-2}
                    & Se analiza las diferencias entre las gráficas. \\ \hline
                \multirow{3}{4.5cm}{Análisis del espectro de magnitud cuando se agregan ceros}
                    & Se dibuja y se analiza el espectro de magnitud con periodo $T_r=4$, $N=100$ y periodo de ceros $T_0=0$. \\ \cline{2-2}
                    & Se dibuja y se analiza el espectro de magnitud con el mismo $N$, $T_r$ y 3 valores diferentes de periodo de ceros $T_0$. \\ \cline{2-2}
                    & Se analiza las diferencias entre las gráficas. \\ \hline
                \multirow{2}{4.5cm}{¿Número de coeficientes cambian cuando $T$ cambia?}
                    & Se utilizan los resultados del objetivo 3 para analizar la igualdad de Parseval en cada caso. \\ \cline{2-2}
                    & Se concluye si el valor de armónicos necesarios para tener una reconstrucción aceptable de la señal cambia o se mantiene. \\ \hline
                \multirow{2}{4.5cm}{¿Número de coeficientes cambian cuando se agregan ceros?}
                    & Se utilizan los resultados del objetivo 4 para analizar la igualdad de Parseval en cada caso. \\ \cline{2-2}
                    & Se concluye si el valor de armónicos necesarios para tener una reconstrucción aceptable de la señal cambia o se mantiene. \\ \hline
            \end{tabular}
        \end{center}
        \caption{Plan de pruebas.}
    \end{table}

\section*{Análisis de Resultados}
    Para la obtención de los resultados se realizaron dos scripts en MATLAB. En el primer script se 
    realizó el calculo de los coefientes de Fourier de manera manual y se transcribieron los 
    resultados al código. En dicha simulación, los coeficientes $A_n$ cuando $n$ es mayor que 0, se 
    cancelaban. Esta primera simulación resultaba valida para reconstruir la señal original a partir de la 
    expresión de la serie de Fourier, sin embargo, evaluar los diferentes escenarios de pruebas 
    planteadas en el trabajo como agregar espacios en blanco o modificar el periodo de la señal 
    requeriría de cambios drásticos en la escritura y evaluación del código.\\
    \\
    El anterior escenario fue motivación para realizar una segunda versión del script, que permitiera 
    generalizar los escenarios planteados en el trabajo y permitiera alcanzar conclusiones completas 
    del mismo.\\
    \\
    El segundo código realiza los cálculos de los coeficientes de manera simbólica, los cuales luego son 
    evaluados en vectores numéricos aumentando así su velocidad de procesamiento. El límite 
    practico en tiempo de ejecución se encontró en un valor cercano a los 5000 armónicos. Sin 
    embargo, debido a que no es necesaria tal cantidad de cálculos, se acotó el máximo posible de 
    armónicos a 100, los cuales permiten de manera suficiente alcanzar los objetivos esperados del 
    trabajo.\\
    \\
    \textbf{Desarrollo del Objetivo Clave 1--Reconstrucción de la señal}

    \begin{figure}[H]
        \centering 
        \begin{subfigure}[h]{0.45\linewidth}
            \includegraphics[width=\linewidth]{img/figure2_A.png}
            \caption{Gráfica original.}
            \label{figure2_A}
        \end{subfigure}
        \begin{subfigure}[h]{0.45\linewidth}
            \includegraphics[width=\linewidth]{img/figure2_B.png}
            \caption{Gráfica reconstruida.}
            \label{figure2_B}
        \end{subfigure}
        \caption{Gráficas del primer script.}
        \label{figure2}
    \end{figure}

    \begin{figure}[H]
        \centering
        \includegraphics[width=0.5\linewidth]{img/figure3.png}
        \caption{Gráfica sobrepuestas de la señal original y reconstruida.}
        \label{figure3}
    \end{figure}

    En las figuras \ref{figure2_A}, \ref{figure2_B} y \ref{figure3} se puede observar el principal resultado visible de esta simulación: Al 
    desarrollar computacionalmente la sumatoria de la serie de Fourier se obtiene como resultado 
    una señal gráficamente muy similar a la señal original. Se observan similitudes en las regiones 
    continuas del diente de sierra o rampa. Sin embargo, como es de esperarse, las señales no son 
    completamente idénticas, ya que en los puntos de discontinuidad de la señal periódica original se 
    generan picos de amplitud explicados por el fenómeno de Gibbs. Las figuras anteriormente 
    expuestas corresponden a una simulación realizada con la sumatoria de los 100 primeros 
    armónicos. Mas adelante se abordarán escenarios de simulación con diferentes cantidades de 
    armónicos en la serie.\\
    \\
    \textbf{Desarrollo del Objetivo Clave 2--Justificación del Numero de Coeficientes}

    \begin{figure}[H]
        \centering
        \begin{subfigure}[h]{0.45\linewidth}
            \includegraphics[width=\linewidth]{img/figure4_A.png}
            \caption{Gráfica 1 armónico.}
            \label{figure4_A}
        \end{subfigure}
        \begin{subfigure}[h]{0.45\linewidth}
            \includegraphics[width=\linewidth]{img/figure4_B.png}
            \caption{Gráfica 10 armónicos.}
            \label{figure4_B}
        \end{subfigure}
        \begin{subfigure}[h]{0.45\linewidth}
            \includegraphics[width=\linewidth]{img/figure4_C.png}
            \caption{Gráfica 50 armónicos.}
            \label{figure4_C}
        \end{subfigure}
        \begin{subfigure}[h]{0.45\linewidth}
            \includegraphics[width=\linewidth]{img/figure4_D.png}
            \caption{Gráfica 100 armónicos.}
            \label{figure4_D}
        \end{subfigure}
        \caption{Gráficas sobrepuestas de las señales con diferente número de armónicos.}
        \label{figure4}
    \end{figure}

    Para justificar la cantidad de coeficientes o armónicos que son necesarios para reconstruir está 
    señal de manera adecuada, se usaran dos criterios diferentes: \textbf{La igualdad de Parseval} y el 
    \textbf{fenómeno de Gibbs evaluado en el punto de la discontinuidad}.\\
    \\
    Se considerará que la cantidad de armónicos es suficiente cuando la relación entre los dos lados 
    de la igualdad de Parseval \textbf{supere el 99\% de similitud}, esto es:

    \begin{equation*}
        \frac{2}{T} \int_{0}^{T} \left|f(t)\right|^2 dt=\frac{a_{0}^2}{2} + \sum_{n=1}^{n=narm} (a_{n}^2 +b_{n}^2)
    \end{equation*}

    Por lo tanto, para encontrar el ratio de similitud proponemos la siguiente expresión:

    \begin{equation*}
        100 \frac{\frac{2}{T} \int_{0}^{T} \left|f(t)\right|^2 dt}{\frac{a_{0}^2}{2} +\sum_{n=1}^{n=narm}(a_{n}^2 +b_{n}^2)} > 99
    \end{equation*}

    Evaluando en la simulación encontramos que en el armónico $narm=15$, el ratio de 
    similitud supera el 99\% ubicándose en 99.0198\%. Esta es la señal reconstruida para ese 
    número de armónicos:
    
    \begin{figure}
       \centering 
       \includegraphics[width=0.9\linewidth]{img/figure5.png}
       \caption{Gráficas sobrepuestas de las señales para 15 armónicos.}
       \label{figure5}
    \end{figure}

    Como segundo criterio de confirmación de la cantidad de armónicos necesarios para la 
    reconstrucción de la señal, evaluamos la vecindad de la discontinuidad en $t=3.5$ para 
    confirmar que el valor de la señal reconstruida en la discontinuidad es igual a el valor 
    medio de la suma de los limites por la izquierda y por la derecha en la señal original, es 
    decir: 

    \begin{equation*}
       Sf(t_0) \cong \frac{f(t_0^+)+f(t_0^-)}{2}
    \end{equation*}

    donde $t_0=3.5$ que es el valor de la discontinuidad. Al evaluar en la simulación 
    confirmamos que:\\
    \\
     $Sf(t_0)=\frac{f(t_0^+)+f(t_0^-)}{2}=0.5000$ por lo que se confirma que el fenómeno de Gibbs 
     en la desigualdad converge a su valor medio. Por lo tanto, verificamos que 15 armónicos 
     son suficientes para tener una reconstrucción aceptable de la señal.

\section*{Conclusiones}
    Aunque puede parecer que gastaron sus mejores ideas en la sección de análisis, la 
    diferencia de las conclusiones es que en este punto ustedes tienen una visión completa 
    del trabajo, ya han abordado todas las fases y por lo tanto están en la capacidad de 
    realizar un compendio de los aprendizajes obtenidos.\\
    Esos aprendizajes deben estar estrechamente relacionados con lo que buscan en el 
    trabajo, con lo que plantearon en el pro qué y el para qué en la introducción. Si esto fuera 
    su trabajo de grado, las conclusiones deben estar relacionadas con los objetivos de ese 
    trabajo.\\
    Lo anterior implica que la siguiente conclusión no es una conclusión válida:\\
    ... MATLAB es un entorno de simulación muy adecuado, ya que permite implementar 
    sistemas de telecomunicaciones...

\begin{thebibliography}{99}
    \bibitem{diapositivas}
        M. Silva, “Capítulo II: Análisis de Fourier,” Notas Cl., pp. 3–70, 2021.
    \bibitem{Sawtooth}
        A. Engineering, M. Subject, L. Transform, and O. F. Periodic, “Snpit \& rc,” p. 11, [Online]. Available: https://www.slideshare.net/surtikaushal/laplace-periodic-function-with-graph.
    \bibitem{parimpar}
        E. Rojero, “Matemáticas Avanzadas,” Univ. Nac. Autónoma México, vol. 0.1, p. 52, 2009, [Online]. Available: https://openlibra.com/es/book/download/matematicas-avanzadas.
\end{thebibliography}
\end{document}
