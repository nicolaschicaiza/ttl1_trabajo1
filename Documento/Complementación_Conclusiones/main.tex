%Preamble
\documentclass[11pt]{article}
\brokenpenalty=10000 
\usepackage[spanish]{babel}
\usepackage[utf8]{inputenc}
\usepackage{times}
\usepackage[T1]{fontenc}
\usepackage{multicol}
\usepackage{tabularx} % extra features for tabular environment
\usepackage{amsmath}  % improve math presentation
\usepackage{graphicx} % takes care of graphic including machinery
\usepackage{geometry} % decreases margins
\usepackage{cite} % takes care of citations
\usepackage[final]{hyperref} % adds hyper links inside the generated pdf file
\usepackage{booktabs}
\usepackage{subcaption}
\usepackage{fancyhdr}
\usepackage{authblk}
\usepackage{parskip}
\usepackage{amssymb, amsmath} % Paquetes matemáticos de la American Mathematical Society
\usepackage{float}
\usepackage{multirow}
\usepackage[all]{xy}
\usepackage{tikz}
\usetikzlibrary{matrix}
\usetikzlibrary{calc}
\usetikzlibrary{fit}
%\usepackage{showframe}

\geometry{
    papersize = {216mm, 279mm},
    width = 19.6cm,
    height = 25cm,
    headsep = 5mm,
    head = 2.8cm,
    marginpar = 5mm,
    includeall,
}

\fancyhf{}
\renewcommand{\headrulewidth}{0pt}
\fancyhead[LO,LE]{
    \begin{minipage}{3cm}
        \includegraphics[width=0.7\textwidth]{Escudo.jpg}
    \end{minipage}
}
\fancyhead[RO,RE]{
    \textsf{
        Correción conclusiones Trabajo 1\\
        Teoría de telecomunicaciones I, Grupo A12\\
        \date{\today}   
    }
}
\fancyfoot[C]{\thepage}

\pagestyle{fancy}

\hypersetup{
	colorlinks=true,       % false: boxed links; true: colored links
	linkcolor=black,        % color of internal links
	citecolor=black,        % color of links to bibliography
	filecolor=magenta,     % color of file links
	urlcolor=blue         
}
\spanishdecimal{.}

%++++++++++++++++++++++++++++++++++++++++
%Content
\usepackage[small,compact]{titlesec}
\titleformat{\subsubsection}[runin]
            {\normalfont\it}
            {\thesubsubsection}{0.5em}{}[:]

\renewcommand*{\Authsep}{, }
\renewcommand*{\Authand}{ y }
\renewcommand*{\Authands}{, }
\renewcommand*{\Affilfont}{\normalsize}
%\renewcommand*{\Authfont}{\bfseries}    % make author names boldface    
\setlength{\affilsep}{0.5em}   % set the space between author and affiliation

\title{
    \fontsize{26}{26}\selectfont 
    \textbf{Series de Fourier}}

\author[1]{Jefry Nicolás Chicaiza}
\author[2]{Jose Nicolás Zambrano}
\affil[1]{jefryn@unicauca.edu.co}
\affil[2]{jnzambranob@unicauca.edu.co}
\date{}

\begin{document}
\maketitle
\thispagestyle{fancy}
\begin{abstract}
    Este documento presenta la corrección de las conclusiones del Trabajo 1, se agregan 
    las conclusiones validas y las conclusiones que se destacan con las correcciones que 
    se le hicieron al trabajo en general mencionadas por el tutor de la asignatura.       
\end{abstract}
\begin{multicols}{2}
\section*{Conclusiones en el informe entregado}
    Se concluye que una adecuada optimización del código es necesaria para la evaluación de cantidades 
    considerables de armónicos. Una inadecuada implementación del código puede llevar a hacer inviable 
    la simulación en situaciones de múltiples repeticiones. Además el uso de cálculos numéricos en vez 
    de cálculos simbólicos acelera el tiempo de ejecución del algoritmo de simulación.
    
    Un adecuado plan de pruebas es necesario antes de iniciar con la construcción de un algoritmo 
    de simulación. Esto debido a que si se inicia construyendo primero la simulación, se corre el 
    riesgo de perder trabajo debido a un inadecuado planteamiento del algoritmo.
    
    La igualdad de Parseval es una herramienta simple y adecuada para evaluar si la reconstrucción de 
    una señal a partir de la serie de Fourier es adecuada o no. Por otro lado, el criterio de Gibbs 
    presenta dificultades para su evaluación en sistemas de tiempo discreto como lo es una simulación 
    computacional. Esto debido a la incertidumbre generada en los puntos de desigualdad y al error de 
    cuantificación que se introduce debido a ello.

\section*{Conclusiones de corrección}
    La realización de los cálculos analíticos antes de realizar una simulación de un algoritmo de bajo
    o alto nivel, es importante para conocer los resultados a los que se quiere llegar. Además el uso 
    de esta comparación se presta como una herramienta para comprender de forma más precisa lo que 
    ocurre al aplicar los diferentes teoremas del documento guiá del trabajo.

    El espectro de magnitud de una señal con un número finito de armónicos, al modificar su periodo, 
    el número de componentes suficiente para reconstruir la señal y la intensidad de cada componente 
    se mantiene. Sin embargo, dada la proporción inversa que se tiene analíticamente entre el periodo
    y la frecuencia, se aprecia que el aumento del periodo de la señal provoca que la frecuencia 
    tienda a valores cercanos a cero.

    El aumento del periodo de una señal con segmentos sin amplitud provoca un mayor número de componentes 
    en frecuencia, que a su vez genera cambios en el espectro de magnitud con una reducción en la intensidad 
    de la señal. Aunque la forma de la envolvente que describe el espectro de magnitud no
    cambia, es posible apreciar esta con mas precisión. 


\end{multicols}
\end{document}
